\documentclass[UTF8]{article}
\usepackage{graphicx}
\usepackage{subfigure}
\usepackage{amsmath}
\usepackage{makecell}
\usepackage[utf8]{inputenc}
\usepackage[space]{ctex} %中文包
\usepackage{listings} %放代码
\usepackage{xcolor} %代码着色宏包
\usepackage{CJK} %显示中文宏包
\usepackage{float}
\usepackage{diagbox}
\usepackage{bm}
\usepackage{ulem} 
\usepackage{amssymb}
\usepackage{soul}
\usepackage{color}
\usepackage{geometry}
\usepackage{fancybox} %花里胡哨的盒子
\usepackage{xhfill} %填充包, 可画分割线 https://www.latexstudio.net/archives/8245
\usepackage{multicol} %多栏包
\usepackage{enumerate} %可以方便地自定义枚举标题
\usepackage{multirow} %表格中多行单元格合并
\usepackage{wasysym} %可以使用wasysym里的一堆奇奇怪怪的符号
\usepackage{hyperref} % url
%%%%%%%%%%%%%%%伪代码%%%%%%%%%%%%%%%
\usepackage{amsmath}
\usepackage{algorithm}
\usepackage[noend]{algpseudocode}
%%%%%%%%%%%%%%%画图包%%%%%%%%%%%%%%%
\usepackage{tikz}
\usepackage{pgfplots} % http://pgfplots.sourceforge.net/gallery.html
\usetikzlibrary{pgfplots.patchplots} % 拟合支持
\usetikzlibrary{arrows,shapes,automata,petri,positioning,calc} % 状态图支持
\usetikzlibrary{arrows.meta} % 箭头
\usetikzlibrary{shadows} % 阴影支持
\usepackage{forest} % 画树

\geometry{left = 1.5cm, right = 1.5cm, top=1.5cm, bottom=2cm}

\definecolor{mygreen}{rgb}{0,0.6,0}
\definecolor{mygray}{rgb}{0.5,0.5,0.5}
\definecolor{mymauve}{rgb}{0.58,0,0.82}
\lstset{
	backgroundcolor=\color{white}, 
	%\tiny < \scriptsize < \footnotesize < \small < \normalsize < \large < \Large < \LARGE < \huge < \Huge
	basicstyle = \footnotesize,       
	breakatwhitespace = false,        
	breaklines = true,                 
	captionpos = b,                    
	commentstyle = \color{mygreen}\bfseries,
	extendedchars = false,
	frame = shadowbox, 
	framerule=0.5pt,
	keepspaces=true,
	keywordstyle=\color{blue}\bfseries, % keyword style
	language = C++,                     % the language of code
	otherkeywords={string}, 
	numbers=left, 
	numbersep=5pt,
	numberstyle=\tiny\color{mygray},
	rulecolor=\color{black},         
	showspaces=false,  
	showstringspaces=false, 
	showtabs=false,    
	stepnumber=1,         
	stringstyle=\color{mymauve},        % string literal style
	tabsize=4,          
	title=\lstname           
}

%\sum\nolimits_{j=1}^{M}   上下标位于求和符号的水平右端,
%\sum\limits_{j=1}^{M}   上下标位于求和符号的上下处,
%\sum_{j=1}^{M}  对上下标位置没有设定,会随公式所处环境自动调整。

%%%%%%%%%%%%%画图包%%%%%%%%%%%%%
\usepackage{tikz}
%%%%%%%%%%%%%好看的矩形%%%%%%%%%%%%%
\tikzset{
  rect1/.style = {
    shape = rectangle,% 指定样式
    minimum height=2cm,% 最小高度
    minimum width=4cm,% 最小宽度
    align = center,% 文字居中
    drop shadow,% 阴影
  }
}
%%%%%%%%%%%%%画图背景包%%%%%%%%%%%%%
\usetikzlibrary{backgrounds}

%%%%%%%%%%%%%在tikz中画一个顶点%%%%%%%%%%%%%
%%%%%%%%%%%%%#1:node名称%%%%%%%%%%%%%
%%%%%%%%%%%%%#2:位置%%%%%%%%%%%%%
%%%%%%%%%%%%%#3:标签%%%%%%%%%%%%%
\newcommand{\newVertex}[3]{\node[circle, draw=black, line width=1pt, scale=0.8] (#1) at #2{#3}}
%%%%%%%%%%%%%在tikz中画一条边%%%%%%%%%%%%%
\newcommand{\newEdge}[2]{\draw [black,very thick](#1)--(#2)}
%%%%%%%%%%%%%在tikz中放一个标签%%%%%%%%%%%%%
%%%%%%%%%%%%%#1:名称%%%%%%%%%%%%%
%%%%%%%%%%%%%#2:位置%%%%%%%%%%%%%
%%%%%%%%%%%%%#3:标签内容%%%%%%%%%%%%%
\newcommand{\newLabel}[3]{\node[line width=1pt] (#1) at #2{#3}}

%%%%%%%%%%%%%强制跳过一行%%%%%%%%%%%%%
\newcommand{\jumpLine} {\hspace*{\fill} \par}
%%%%%%%%%%%%%关键点指令,可用itemise替代%%%%%%%%%%%%%
\newcommand{\average}[1]{\left\langle #1\right\rangle }
%%%%%%%%%%%%%表格内嵌套表格%%%%%%%%%%%%%
\newcommand{\keypoint}[2]{$\bullet$\textbf{#1}\quad#2\par}
%%%%%%%%%%%%%<T>平均值表示%%%%%%%%%%%%%
\newcommand{\tabincell}[2]{\begin{tabular}{@{}#1@{}}#2\end{tabular}}%放在导言区
%%%%%%%%%%%%%大黑点item头%%%%%%%%%%%%%
\newcommand{\itemblt}{\item[$\bullet$]}
%%%%%%%%%%%%%大圈item头%%%%%%%%%%%%%
\newcommand{\itemc}{\item[$\circ$]}
%%%%%%%%%%%%%大星星item头%%%%%%%%%%%%%
\newcommand{\itembs}{\item[$\bigstar$]}
%%%%%%%%%%%%%右▷item头%%%%%%%%%%%%%
\newcommand{\itemrhd}{\item[$\rhd$]}
%%%%%%%%%%%%%定义为%%%%%%%%%%%%%
\newcommand{\defas}{=_{df}}
%%%%%%%%%%%%%偏导%%%%%%%%%%%%%
\newcommand{\partialx}[2]{\frac{\partial #1}{\partial #2}}
%%%%%%%%%%%%%蕴含%%%%%%%%%%%%%
\newcommand{\imp}{\rightarrow}
%%%%%%%%%%%%%上取整%%%%%%%%%%%%%
\newcommand{\ceil}[1]{\lceil#1\rceil}
%%%%%%%%%%%%%下取整%%%%%%%%%%%%%
\newcommand{\floor}[1]{\lfloor#1\rfloor}
%%%%%%%%%%%%%textbullet%%%%%%%%%%%%%
\newcommand{\blt}{\bullet}

%%%%%%%%%%%%%双线分割线%%%%%%%%%%%%%
\newcommand*{\doublerule}{\hrule width \hsize height 1pt \kern 0.5mm \hrule width \hsize height 2pt}
%%%%%%%%%%%%%双线中间可加东西的分割线%%%%%%%%%%%%%
\newcommand\doublerulefill{\leavevmode\leaders\vbox{\hrule width .1pt\kern1pt\hrule}\hfill\kern0pt }
%%%%%%%%%%%%%左大括号%%%%%%%%%%%%%
\newcommand{\leftbig}[1]{\left\{\begin{array}{l}#1\end{array}\right.}
%%%%%%%%%%%%%矩阵%%%%%%%%%%%%%
\newcommand{\mat}[2]{\left[\begin{array}{#1}#2\end{array}\right]}
%%%%%%%%%%%%%可换行圆角文本框%%%%%%%%%%%%%
\newcommand{\ovalboxn}[1]{\ovalbox{\tabincell{l}{#1}}}
%%%%%%%%%%%%%设置section的counter, 使从1开始%%%%%%%%%%%%%
\setcounter{section}{0}

%%%%%%%%%%%%%Colors%%%%%%%%%%%%%
\newcommand{\lightercolor}[3]{% Reference Color, Percentage, New Color Name
    \colorlet{#3}{#1!#2!white}
}
\newcommand{\darkercolor}[3]{% Reference Color, Percentage, New Color Name
    \colorlet{#3}{#1!#2!black}
}
\definecolor{aquamarine}{rgb}{0.5, 1.0, 0.83}
\definecolor{Seashell}{RGB}{255, 245, 238} %背景色浅一点的
\definecolor{Firebrick4}{RGB}{255, 0, 0}%文字颜色红一点的
\lightercolor{gray}{20}{lgray}
\newcommand{\hlg}[1]{
	\begingroup
		\sethlcolor{lgray}%背景色
		\textcolor{black}{\hl{\mbox{#1}}}%textcolor里面对应文字颜色
	\endgroup
}



\title{编译原理与技术 H4-1}
\date{}
\author{PB18111697 王章瀚}

\begin{document}
\maketitle
\section*{3.19}
\noindent 考虑下面的文法
$$\begin{array}{l}
E\rightarrow E+T|T\\
T\rightarrow TF|F\\
F\rightarrow F*|a|b
\end{array}$$
\subsection*{(a). 为此文法构造SLR分析表}
\noindent 为了构造SLR分析表, 首先需要求出其LR(0)项目集规范族.\\
\begin{itemize}
\item 首先写出拓广的表达式文法
$$\begin{array}{ll}
 & E'\rightarrow E\\
1 & E\rightarrow E+T\\
2 & E\rightarrow T\\
3 & T\rightarrow TF\\
4 & T\rightarrow F\\
5 & F\rightarrow F*\\
6 & F\rightarrow a\\
7 & F\rightarrow b
\end{array}$$
\item 求 $Closure(\{[E'\rightarrow E]\})$\\
$I_0=Closure(\{[E'\rightarrow E]\})$应含有:
$$I_0\left\{\begin{array}{ll}
E'\rightarrow \blt E & T\rightarrow\blt F \\
E\rightarrow \blt E+T & F\rightarrow\blt F* \\
E\rightarrow \blt T & F\rightarrow\blt a \\
T\rightarrow \blt TF & F\rightarrow\blt b \\
\end{array}\right.$$
\item 开始求相应LR(0)项目集规范族\\
	\begin{itemize}
	\item 从$I=I_0$出发:
		\begin{itemize}
		\item 对$X=E$, 得到$I_1$:
		$$I_1\left\{\begin{array}{ll}
		E'\rightarrow E\blt \\
		E\rightarrow E\blt +T \\
		\end{array}\right.$$
		\item 对$X=T$, 得到$I_2$:
		$$I_2\left\{\begin{array}{ll}
		E\rightarrow T\blt & F\rightarrow\blt a \\
		T\rightarrow T\blt F & F\rightarrow\blt b \\
		& F\rightarrow\blt F* \\
		\end{array}\right.$$
		\item 对$X=F$, 得到$I_3$:
		$$I_3\left\{\begin{array}{ll}
		T\rightarrow F\blt\\
		F\rightarrow F\blt * \\
		\end{array}\right.$$
		\item 对$X=a$或$X=a$, 得到$I_4$, $I_5$:
		$$\begin{array}{ll}
		I_4: & \left\{F\rightarrow a\blt \right.\\
		\\
		I_5: & \left\{ F\rightarrow b\blt \right.\\
		\end{array}$$
		\end{itemize}
	\item 从$I_1, I_2, I_3$出发:
		\begin{itemize}
		\item 从$I_1$出发, 若$X=+$, 则能得到$I_6$
			$$I_6\left\{\begin{array}{ll}
			E\rightarrow E+\blt T & F\rightarrow\blt F* \\
			T\rightarrow\blt TF & F\rightarrow\blt a \\
			T\rightarrow\blt F & F\rightarrow\blt b \\
			\end{array}\right.$$
		\item 从$I_2$出发, 
			\begin{itemize}
			\item 若$X=F$, 得到$I_7$:
				$$I_7\left\{\begin{array}{ll}
				T\rightarrow TF\blt \\
				F\rightarrow F\blt * \\
				\end{array}\right.$$
			\item 若$X=a$或$X=b$, 分别得到$I_6, I_7$
			\end{itemize}
		\item 从$I_3$出发, $X=*$时得到$I_8$:
			$$I_8\left\{\begin{array}{ll}
			F\rightarrow F *\blt \\
			\end{array}\right.$$
		\end{itemize}
	\item 从$I_7, I_8$出发均已不能得到新的项目集, 从$I_6$出发, 若$X=T$, 得到$I_9$:
		$$I_9\left\{\begin{array}{ll}
		E\rightarrow E+T\blt & F\rightarrow\blt a \\
		T\rightarrow T\blt F & F\rightarrow\blt b \\
		& F\rightarrow\blt F* \\
		\end{array}\right.$$
	\end{itemize}

\item 至此, 完成了该文法的LR(0)项目集规范族的构造, 下一步画出其对应的DFA转换图\\
	\begin{tikzpicture}[shorten >=1pt,node distance=2cm,auto]
	\tikzstyle{every state}=[fill={rgb:black,1;white,10}]
	
	% 从I0出发
	\node[state](i0) at(0,0) {$I_0$};
	\node[state](i1) at(2,4){$I_1$};
	\node[state](i2) at(2,0){$I_2$};
	\node[state](i3) at(4,-4){$I_3$};
	\node[state](i4) at(4,0){$I_4$};
	\node[state](i5) at(4,-2){$I_5$};
	\node[state](i6) at(8,4) {$I_6$};
	\node[state](i7) at(4,2){$I_7$};
	\node[state](i8) at(6,-4){$I_8$};
	\node[state](i9) at(10,4){$I_9$};
	
	\path[->] (i0) edge node{$E$} (i1);
	\path[->] (i0) edge node{$T$} (i2);
	\path[->] (i0) edge[bend right] node{$F$} (i3);
	
	\node(n4) at(-2,2){指向$I_4$};
	\node(n5) at(-2,-2){指向$I_5$};
	\path[->] (i0) edge node{$a$} (n4);
	\path[->] (i0) edge node{$b$} (n5);
	
	% 从I1出发
	\path[->] (i1) edge node{$+$} (i6);
	
	% 从I2出发
	\path[->] (i2) edge node{$F$} (i7);
	\path[->] (i2) edge node{$a$} (i4);
	\path[->] (i2) edge node{$b$} (i5);
	
	% 从I3出发
	\path[->] (i3) edge node{$*$} (i8);
	
	% 从I4出发
	\path[->] (i6) edge node{$T$} (i9);
	\path[->] (i6) edge[bend left] node{$F$} (i3);
	\path[->] (i6) edge[bend left] node{$a$} (i4);
	\path[->] (i6) edge[bend left] node{$b$} (i5);
	
	% 从I5出发
	\node(n0) at(6,2){指向$I_8$};
	\path[->] (i7) edge node{$*$} (n0);
	
	% 从I9出发
	\node(n1) at(12,6){指向$I_7$};
	\node(n2) at(12,4){指向$I_4$};
	\node(n3) at(12,2){指向$I_5$};
	\path[->] (i9) edge node{$F$} (n1);
	\path[->] (i9) edge node{$a$} (n2);
	\path[->] (i9) edge node{$b$} (n3);
	\end{tikzpicture}
\item 之后就可以直接构造表了:
	\begin{enumerate}[(1) ]
	\item 前面已经构造了项目集规范族
	\item 构造action函数, 这里用表的形式写出. 此前, 先给出各个FOLLOW
	$$\begin{array}{l}
	FOLLOW(E)=\{+,\$\}\\
	FOLLOW(T)=\{a,b,+,\$\}\\
	FOLLOW(F)=\{a,b,*,+,\$\}\\
	\end{array}$$
	\begin{center}
	\begin{tabular}{|c|c|c|c|c|c|}
	\hline
	 & a & b & + & * & \$\\
 	\hline
 	$I_0$ & s4 & s5 &  &  &  \\
	\hline
	$I_1$ &  &  & s6 &  & acc \\
	\hline
	$I_2$ & s4 & s5 & r2 &  & r2 \\
	\hline
	$I_3$ & r4 & r4 & r4 & s8  & r4 \\
	\hline
	$I_4$ & r6 & r6 & r6 & r6 & r6 \\
	\hline
	$I_5$ & r7 & r7 & r7 & r7 & r7 \\
	\hline
	$I_6$ & s4 & s5 &  &  &  \\
	\hline
	$I_7$ & r3 & r3 & r3 & s8 & r3 \\
	\hline
	$I_8$ & r5 & r5 & r5 & r5 & r5 \\
	\hline
	$I_9$ & s4 & s5 & r1 &  & r1 \\
	\hline
	\end{tabular}
	\end{center}
	\end{enumerate}
\item 构造goto函数
	\begin{center}
	\begin{tabular}{|c|c|c|c|}
	\hline
	 & E & T & F \\
	\hline
	$I_0$ & 1 & 2 & 3 \\
	\hline
	$I_1$ &  &  &  \\
	\hline
	$I_2$ &  &  & 7 \\
	\hline
	$I_3$ &  &  &  \\
	\hline
	$I_4$ &  &  &  \\
	\hline
	$I_5$ &  &  &  \\
	\hline
	$I_6$ &  & 9 & 3 \\
	\hline
	$I_7$ &  &  &  \\
	\hline
	$I_8$ &  &  &  \\
	\hline
	$I_9$ &  &  & 7 \\
	\hline
	\end{tabular}
	\end{center}
\item 至此, 该文法的SLR(1)分析表构造完毕
\hfill$\square$
\end{itemize}

\subsection*{(b). 为此文法构造LALR分析表}
\begin{itemize}
\item 首先写出拓广的表达式文法
$$\begin{array}{ll}
 & E'\rightarrow E\\
1 & E\rightarrow E+T\\
2 & E\rightarrow T\\
3 & T\rightarrow TF\\
4 & T\rightarrow F\\
5 & F\rightarrow F*\\
6 & F\rightarrow a\\
7 & F\rightarrow b
\end{array}$$
\item 然后构造LR(1)项目集规范族, 顺便求出action和goto函数
	\begin{itemize}
	\item 先写出$I_0$
	$$I_0\left\{\begin{array}{ll}
	E'\rightarrow \blt E & ,\$\\
	E\rightarrow \blt E+T & ,+/\$\\
	E\rightarrow\blt T & ,+/\$\\
	T\rightarrow\blt TF & ,a/b/+/\$\\
	T\rightarrow\blt F & , a/b/+/\$\\
	F\rightarrow\blt F* & , a/b/+/*/\$\\
	F\rightarrow\blt a & ,a/b/+/*/\$\\
	F\rightarrow\blt b & ,a/b/+/*/\$\\
	\end{array}\right.$$
	\item 从$I_0$出发,
		\begin{itemize}
		\item 考虑$goto(I_0,E)$
			$$I_1\left\{\begin{array}{ll}
			E'\rightarrow E\blt & ,\$\\
			E\rightarrow  E\blt+T & ,+/\$\\
			\end{array}\right.$$
		\item 考虑$goto(I_0,T)$
			$$I_2\left\{\begin{array}{ll}
			E\rightarrow T\blt & ,+/\$\\
			T\rightarrow T\blt F & ,a/b/+/\$\\
			F\rightarrow\blt F* & , a/b/+/*/\$\\
			F\rightarrow\blt a & ,a/b/+/*/\$\\
			F\rightarrow\blt b & ,a/b/+/*/\$\\
			\end{array}\right.$$
		\item 考虑$goto(I_0,F)$
			$$I_3\left\{\begin{array}{ll}
			T\rightarrow F\blt & , a/b/+/\$\\
			F\rightarrow F\blt * & , a/b/+/*/\$\\
			\end{array}\right.$$
		\item 考虑$goto(I_0,a)$
			$$I_4\left\{\begin{array}{ll}
			F\rightarrow a\blt & ,a/b/+/*/\$\\
			\end{array}\right.$$
		\item 考虑$goto(I_0,a)$
			$$I_5\left\{\begin{array}{ll}
			F\rightarrow b\blt & ,a/b/+/*/\$\\
			\end{array}\right.$$
		\end{itemize}
	\item 从$I_1$出发, 考虑$goto(I_1,+)$
		$$I_6\left\{\begin{array}{ll}
		E\rightarrow  E+\blt T & ,+/\$\\
		T\rightarrow\blt TF & ,a/b/+/\$\\
		T\rightarrow\blt F & , a/b/+/\$\\
		F\rightarrow\blt F* & , a/b/+/*/\$\\
		F\rightarrow\blt a & ,a/b/+/*/\$\\
		F\rightarrow\blt b & ,a/b/+/*/\$\\
		\end{array}\right.$$
	\item 从$I_2$出发
		\begin{itemize}
		\item 考虑$goto(I_2,F)$
			$$I_7\left\{\begin{array}{ll}
			T\rightarrow TF\blt & ,a/b/+/\$\\
			F\rightarrow F\blt * & , a/b/+/*/\$\\
			\end{array}\right.$$
		\item 考虑$goto(I_2,a)$, 得到$I_4$
		\item 考虑$goto(I_2,b)$, 得到$I_5$
		\end{itemize}
	\item 从$I_3$出发, 考虑$goto(I_3,*)$, 得到
		$$I_8\left\{\begin{array}{ll}
		F\rightarrow F*\blt & , a/b/+/*/\$\\
		\end{array}\right.$$
	\item 从$I_4, I_5$无法继续了
	\item 从$I_6$出发
		\begin{itemize}
		\item 考虑$goto(I_6,T)$, 得到
			$$I_9\left\{\begin{array}{ll}
			E\rightarrow  E+T\blt & ,+/\$\\
			T\rightarrow T\blt F & ,a/b/+/\$\\
			F\rightarrow\blt F* & , a/b/+/*/\$\\
			F\rightarrow\blt a & ,a/b/+/*/\$\\
			F\rightarrow\blt b & ,a/b/+/*/\$\\
			\end{array}\right.$$
		\item 考虑$goto(I_6,F)$, 得到$I_3$
		\item 考虑$goto(I_6,a)$和$goto(I_6,b)$, 分别得到$I_4, I_5$
		\end{itemize}
	\item 从$I_7$出发, 考虑$goto(I_7,*)$, 得到$I_8$
	\item 从$I_8$莫得出发
	\item 从$I_9$
		\begin{itemize}
		\item 考虑$goto(I_9,F)$, 得到$I_7$
		\item 考虑$goto(I_9,a), goto(I_9,b)$分别得到$I_4,I_5$
		\end{itemize}
	\end{itemize}
\item 至此, 已经列出所有结果, 尝试寻找同心的LR(1)项目集, 但没有.
\item 开始构造action函数. 如下:
	\begin{center}
	\begin{tabular}{|c|c|c|c|c|c|}
	\hline
	 & a & b & + & * & \$\\
 	\hline
 	$I_0$ & s4 & s5 &  &  &  \\
	\hline
	$I_1$ &  &  & s6 &  & acc \\
	\hline
	$I_2$ & s4 & s5 & r2 &  & r2 \\
	\hline
	$I_3$ & r4 & r4 & r4 & s8  & r4 \\
	\hline
	$I_4$ & r6 & r6 & r6 & r6 & r6 \\
	\hline
	$I_5$ & r7 & r7 & r7 & r7 & r7 \\
	\hline
	$I_6$ & s4 & s5 &  &  &  \\
	\hline
	$I_7$ & r3 & r3 & r3 & s8 & r3 \\
	\hline
	$I_8$ & r5 & r5 & r5 & r5 & r5 \\
	\hline
	$I_9$ & s4 & s5 & r1 &  & r1 \\
	\hline
	\end{tabular}
	\end{center}
\item 构造goto函数
	\begin{center}
	\begin{tabular}{|c|c|c|c|}
	\hline
 & E & T & F \\
	\hline
	$I_0$ & 1 & 2 & 3 \\
	\hline
	$I_1$ &  &  &  \\
	\hline
	$I_2$ &  &  & 7 \\
	\hline
	$I_3$ &  &  &  \\
	\hline
	$I_4$ &  &  &  \\
	\hline
	$I_5$ &  &  &  \\
	\hline
	$I_6$ &  & 9 & 3 \\
	\hline
	$I_7$ &  &  &  \\
	\hline
	$I_8$ &  &  &  \\
	\hline
	$I_9$ &  &  & 7 \\
	\hline
	\end{tabular}
	\end{center}
\item 至此, 该文法的LALR分析表构造完毕.
\end{itemize}


\newpage
\section*{3.20}
\noindent 证明下面的文法
$$\begin{array}{l}
S\rightarrow SA|A\\
A\rightarrow a
\end{array}$$
是SLR(1)文法, 但不是LL(1)文法.\\
\subsection*{1. 证明是SLR(1)文法}
按照前面的方法尝试构造分析表.
\begin{itemize}
\item 拓广文法
	$$\begin{array}{ll}
	  & S'\rightarrow S\\
	1 & S\rightarrow SA\\
	2 & S\rightarrow A\\
	3 & A\rightarrow a
	\end{array}$$
\item 项目集规范族:
	$$\begin{array}{ll}
	I_0:
		\left\{\begin{array}{l}
		S'\rightarrow \blt S\\
		S\rightarrow \blt SA\\
		S\rightarrow\blt A\\
		A\rightarrow \blt a
		\end{array}\right.
	&
	I_1:
		\left\{\begin{array}{l}
		S'\rightarrow S\blt\\
		S\rightarrow S\blt A\\
		A\rightarrow \blt a
		\end{array}\right.
	\\\\
	I_2:
		\left\{\begin{array}{l}
		S\rightarrow A\blt
		\end{array}\right.
	&
	I_3:
		\left\{\begin{array}{l}
		A\rightarrow a\blt
		\end{array}\right.
	\\\\
	I_4:
		\left\{\begin{array}{l}
		S\rightarrow SA\blt
		\end{array}\right.
	\\
	\end{array}$$
\item 构造分析表
	\begin{enumerate}[(1) ]
	\item 构造action函数
		$$\begin{array}{l}
		FOLLOW(S)=\{\$,a\}\\
		FOLLOW(A)=\{\$,a\}
		\end{array}$$
		\begin{center}
		\begin{tabular}{|c|c|c|}
		\hline
		 & a & \$ \\
		\hline
		$I_0$ & s3 &  \\
		\hline
		$I_1$ & s3 & acc \\
		\hline
		$I_2$ & r2 & r2 \\
		\hline
		$I_3$ & r3 & r3 \\
		\hline
		$I_4$ & r1 & r1 \\
		\hline
		\end{tabular}\\
		\end{center}
		显然, 我们可以构造出无冲突的动作. 至此就已经可以认为, 该文法是SLR(1)的了. 为了练手, 再求以下goto函数
		\item 构造goto函数如右
		\begin{tabular}{|c|c|c|}
		\hline
		 & S & A \\
		\hline
		$I_0$ & 1 & 2 \\
		\hline
		$I_1$ &  & 4 \\
		\hline
		$I_2$ &  &  \\
		\hline
		$I_3$ &  &  \\
		\hline
		$I_4$ &  &  \\
		\hline
		\end{tabular}
		\item 至此完整构造出了分析表
	\end{enumerate}
\end{itemize}
\textbf{综上所述, 该文法是SLR(1)的}.

\subsection*{2. 证明不是LL(1)文法}
它显然不是LL(1)文法, 因为LL(1)文法必然不含有左递归, 而$S\rightarrow SA$是左递归的.


\end{document}





