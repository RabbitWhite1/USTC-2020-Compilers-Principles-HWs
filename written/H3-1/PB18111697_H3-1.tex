\documentclass[UTF8]{article}
\usepackage{graphicx}
\usepackage{subfigure}
\usepackage{amsmath}
\usepackage{makecell}
\usepackage[utf8]{inputenc}
\usepackage[space]{ctex} %中文包
\usepackage{listings} %放代码
\usepackage{xcolor} %代码着色宏包
\usepackage{CJK} %显示中文宏包
\usepackage{float}
\usepackage{makecell}
\usepackage{diagbox}
\usepackage{bm}
\usepackage{ulem} 
\usepackage{amssymb}
\usepackage{soul}
\usepackage{color}
\usepackage{geometry}
\usepackage{fancybox} %花里胡哨的盒子
\usepackage{xhfill} %填充包, 可画分割线 https://www.latexstudio.net/archives/8245
\usepackage{multicol} %多栏包
\usepackage{enumerate} %可以方便地自定义枚举标题
\usepackage{multirow} %表格中多行单元格合并
\usepackage{wasysym} %可以使用wasysym里的一堆奇奇怪怪的符号
%%%%%%%%%%%%%%%伪代码%%%%%%%%%%%%%%%
\usepackage{amsmath}
\usepackage{algorithm}
\usepackage[noend]{algpseudocode}
%%%%%%%%%%%%%%%画图包%%%%%%%%%%%%%%%
\usepackage{tikz}
\usepackage{tikZ-timing} % 时序图支持
\usepackage{pgfplots} % http://pgfplots.sourceforge.net/gallery.html
\usetikzlibrary{pgfplots.patchplots} % 拟合支持
\usetikzlibrary{arrows,shapes,automata,petri,positioning,calc} % 状态图支持
\usetikzlibrary{shadows} % 阴影支持
\usepackage{forest} % 画树

\geometry{left = 1.5cm, right = 1.5cm, top=1.5cm, bottom=2cm}

\definecolor{mygreen}{rgb}{0,0.6,0}
\definecolor{mygray}{rgb}{0.5,0.5,0.5}
\definecolor{mymauve}{rgb}{0.58,0,0.82}
\lstset{
	backgroundcolor=\color{white}, 
	%\tiny < \scriptsize < \footnotesize < \small < \normalsize < \large < \Large < \LARGE < \huge < \Huge
	basicstyle = \footnotesize,       
	breakatwhitespace = false,        
	breaklines = true,                 
	captionpos = b,                    
	commentstyle = \color{mygreen}\bfseries,
	extendedchars = false,
	frame = shadowbox, 
	framerule=0.5pt,
	keepspaces=true,
	keywordstyle=\color{blue}\bfseries, % keyword style
	language = C++,                     % the language of code
	otherkeywords={string}, 
	numbers=left, 
	numbersep=5pt,
	numberstyle=\tiny\color{mygray},
	rulecolor=\color{black},         
	showspaces=false,  
	showstringspaces=false, 
	showtabs=false,    
	stepnumber=1,         
	stringstyle=\color{mymauve},        % string literal style
	tabsize=4,          
	title=\lstname           
}

%\sum\nolimits_{j=1}^{M}   上下标位于求和符号的水平右端,
%\sum\limits_{j=1}^{M}   上下标位于求和符号的上下处,
%\sum_{j=1}^{M}  对上下标位置没有设定,会随公式所处环境自动调整。

%%%%%%%%%%%%%画图包%%%%%%%%%%%%%
\usepackage{tikz}
%%%%%%%%%%%%%好看的矩形%%%%%%%%%%%%%
\tikzset{
  rect1/.style = {
    shape = rectangle,% 指定样式
    minimum height=2cm,% 最小高度
    minimum width=4cm,% 最小宽度
    align = center,% 文字居中
    drop shadow,% 阴影
  }
}
%%%%%%%%%%%%%画图背景包%%%%%%%%%%%%%
\usetikzlibrary{backgrounds}

%%%%%%%%%%%%%在tikz中画一个顶点%%%%%%%%%%%%%
%%%%%%%%%%%%%#1:node名称%%%%%%%%%%%%%
%%%%%%%%%%%%%#2:位置%%%%%%%%%%%%%
%%%%%%%%%%%%%#3:标签%%%%%%%%%%%%%
\newcommand{\newVertex}[3]{\node[circle, draw=black, line width=1pt, scale=0.8] (#1) at #2{#3}}
%%%%%%%%%%%%%在tikz中画一条边%%%%%%%%%%%%%
\newcommand{\newEdge}[2]{\draw [black,very thick](#1)--(#2)}
%%%%%%%%%%%%%在tikz中放一个标签%%%%%%%%%%%%%
%%%%%%%%%%%%%#1:名称%%%%%%%%%%%%%
%%%%%%%%%%%%%#2:位置%%%%%%%%%%%%%
%%%%%%%%%%%%%#3:标签内容%%%%%%%%%%%%%
\newcommand{\newLabel}[3]{\node[line width=1pt] (#1) at #2{#3}}

%%%%%%%%%%%%%强制跳过一行%%%%%%%%%%%%%
\newcommand{\jumpLine} {\hspace*{\fill} \par}
%%%%%%%%%%%%%关键点指令,可用itemise替代%%%%%%%%%%%%%
\newcommand{\average}[1]{\left\langle #1\right\rangle }
%%%%%%%%%%%%%表格内嵌套表格%%%%%%%%%%%%%
\newcommand{\keypoint}[2]{$\bullet$\textbf{#1}\quad#2\par}
%%%%%%%%%%%%%<T>平均值表示%%%%%%%%%%%%%
\newcommand{\tabincell}[2]{\begin{tabular}{@{}#1@{}}#2\end{tabular}}%放在导言区
%%%%%%%%%%%%%大黑点item头%%%%%%%%%%%%%
\newcommand{\itemblt}{\item[$\bullet$]}
%%%%%%%%%%%%%大圈item头%%%%%%%%%%%%%
\newcommand{\itemc}{\item[$\circ$]}
%%%%%%%%%%%%%大星星item头%%%%%%%%%%%%%
\newcommand{\itembs}{\item[$\bigstar$]}
%%%%%%%%%%%%%右▷item头%%%%%%%%%%%%%
\newcommand{\itemrhd}{\item[$\rhd$]}
%%%%%%%%%%%%%定义为%%%%%%%%%%%%%
\newcommand{\defas}{=_{df}}
%%%%%%%%%%%%%蕴含%%%%%%%%%%%%%
\newcommand{\imp}{\rightarrow}

%%%%%%%%%%%%%双线分割线%%%%%%%%%%%%%
\newcommand*{\doublerule}{\hrule width \hsize height 1pt \kern 0.5mm \hrule width \hsize height 2pt}
%%%%%%%%%%%%%双线中间可加东西的分割线%%%%%%%%%%%%%
\newcommand\doublerulefill{\leavevmode\leaders\vbox{\hrule width .1pt\kern1pt\hrule}\hfill\kern0pt }
%%%%%%%%%%%%%左大括号%%%%%%%%%%%%%
\newcommand{\leftbig}[1]{\left\{\begin{array}{l}#1\end{array}\right.}
%%%%%%%%%%%%%矩阵%%%%%%%%%%%%%
\newcommand{\mat}[2]{\left[\begin{array}{#1}#2\end{array}\right]}
%%%%%%%%%%%%%可换行圆角文本框%%%%%%%%%%%%%

\newcommand{\ovalboxn}[1]{\ovalbox{\tabincell{l}{#1}}}
%%%%%%%%%%%%%设置section的counter, 使从1开始%%%%%%%%%%%%%
\setcounter{section}{0}


\title{编译原理与技术 H2}
\date{}
\author{PB18111697 王章瀚}

\begin{document}
\maketitle

\section*{3.2}
\noindent 考虑文法$S\rightarrow aSbS|bSaS|\epsilon$
\begin{enumerate}[(a) ]
\item 为句子$abab$构造两个不同的最左推导, 以此说明该文法是二义的.\\
	构造如下:
	\begin{enumerate}[1. ]
	\item $S \Rightarrow_{lm} aSbS \Rightarrow_{lm} abS \Rightarrow_{lm} abaSbS \Rightarrow_{lm} ababS \Rightarrow_{lm} abab$
	\item $S \Rightarrow_{lm} aSbS \Rightarrow_{lm} abSaSbS \Rightarrow_{lm} abaSbS \Rightarrow_{lm} ababS \Rightarrow_{lm} abab$
	\end{enumerate}
\item 为$abab$构造对应的最右推导.\\
	构造为: $S \Rightarrow_{rm} aSbS \Rightarrow_{rm} aSb \Rightarrow_{rm} abSaSb \Rightarrow_{rm} abSab \Rightarrow_{rm} abab$
\item 为$abab$构造对应的分析树.\\
	以(a)中1示例:\\
	\begin{forest}
	[$S$
		[$a$]
		[$S$
			[$\epsilon$]
		]
		[$b$]
		[$S$
			[$a$]
			[$S$
				[$\epsilon$]
			]
			[$b$]
			[$S$
				[$\epsilon$]
			]
		]
	];
	\end{forest}
\item 这个文法产生的语言是什么?\\
	产生的语言是: \textbf{由同样数目的$\bm{a}$和$\bm{b}$的排列}.
\end{enumerate}

\section*{3.6}
\noindent 为字母表$\Sigma=\{a,b\}$上的下列每个语言设计一个文法, 其中哪些语言是正规的?
\begin{enumerate}
\item[($a$) ] 每个$a$后面至少有一个$b$跟随的所有串.\\
	设计的文法对应四元组为$(\{a,b\},\{S,B\}, S, P)$. 其中产生式的集合P如下: 
	$$\begin{array}{l}
	S \rightarrow aBS|aB\\
	B \rightarrow bB|\epsilon
	\end{array}$$
	这个语言\textbf{是正规的}.因为其产生式满足形式为$A\rightarrow aB$或$A\rightarrow a$
\item[($c$) ] \textcolor{red}{$a$和$b$的个数不相等的所有串.}
\end{enumerate}

\section*{3.8}
\begin{enumerate}
\item[(a) ] 消除习题3.1文法($S\rightarrow (L)|a\quad L\rightarrow L,S|S)$的左递归.\\
	也就是消除下式的左递归:
	$$\begin{array}{l}
	S \rightarrow (L)|a\\
	L \rightarrow L,(L)|L,a|(L)|a
	\end{array}$$
	因此可以改写为:
	$$\begin{array}{l}
	S \rightarrow (L)|a\\
	L \rightarrow (L)L'|aL'\\
	L' \rightarrow ,aL'|,(L)L'|\epsilon]
	\end{array}$$
\end{enumerate}

\section*{3.11}
\noindent 构造下面文法的LL(1)分析表.
\begin{enumerate}
\item $S\rightarrow aBS|bAS|\epsilon$
\item $A\rightarrow bAA|a$
\item $B\rightarrow aBB|b$
\end{enumerate}

\section*{3.12}
\noindent 下面的文法是否为LL(1)文法? 说明理由.
\begin{enumerate}
\item $S \rightarrow AB|PQx$
\item $A \rightarrow xy$
\item $B \rightarrow bc$
\item $P \rightarrow dP|\epsilon$
\item $Q \rightarrow aQ|\epsilon$
\end{enumerate}

\end{document}







