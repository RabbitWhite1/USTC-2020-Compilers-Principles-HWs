\documentclass[UTF8]{article}
\usepackage{graphicx}
\usepackage{subfigure}
\usepackage{amsmath}
\usepackage{makecell}
\usepackage[utf8]{inputenc}
\usepackage[space]{ctex} %中文包
\usepackage{listings} %放代码
\usepackage{xcolor} %代码着色宏包
\usepackage{CJK} %显示中文宏包
\usepackage{float}
\usepackage{diagbox}
\usepackage{bm}
\usepackage{ulem} 
\usepackage{amssymb}
\usepackage{soul}
\usepackage{color}
\usepackage{geometry}
\usepackage{fancybox} %花里胡哨的盒子
\usepackage{xhfill} %填充包, 可画分割线 https://www.latexstudio.net/archives/8245
\usepackage{multicol} %多栏包
\usepackage{enumerate} %可以方便地自定义枚举标题
\usepackage{multirow} %表格中多行单元格合并
\usepackage{wasysym} %可以使用wasysym里的一堆奇奇怪怪的符号
\usepackage{hyperref} % url
%%%%%%%%%%%%%%%伪代码%%%%%%%%%%%%%%%
\usepackage{amsmath}
\usepackage{algorithm}
\usepackage[noend]{algpseudocode}
%%%%%%%%%%%%%%%画图包%%%%%%%%%%%%%%%
\usepackage{tikz}
\usepackage{pgfplots} % http://pgfplots.sourceforge.net/gallery.html
\usetikzlibrary{pgfplots.patchplots} % 拟合支持
\usetikzlibrary{arrows,shapes,automata,petri,positioning,calc} % 状态图支持
\usetikzlibrary{arrows.meta} % 箭头
\usetikzlibrary{shadows} % 阴影支持
\usepackage{forest} % 画树

\geometry{left = 1.5cm, right = 1.5cm, top=1.5cm, bottom=2cm}

\definecolor{mygreen}{rgb}{0,0.6,0}
\definecolor{mygray}{rgb}{0.5,0.5,0.5}
\definecolor{mymauve}{rgb}{0.58,0,0.82}
\lstset{
	backgroundcolor=\color{white}, 
	%\tiny < \scriptsize < \footnotesize < \small < \normalsize < \large < \Large < \LARGE < \huge < \Huge
	basicstyle = \footnotesize,       
	breakatwhitespace = false,        
	breaklines = true,                 
	captionpos = b,                    
	commentstyle = \color{mygreen}\bfseries,
	extendedchars = false,
	frame = shadowbox, 
	framerule=0.5pt,
	keepspaces=true,
	keywordstyle=\color{blue}\bfseries, % keyword style
	language = C++,                     % the language of code
	otherkeywords={string}, 
	numbers=left, 
	numbersep=5pt,
	numberstyle=\tiny\color{mygray},
	rulecolor=\color{black},         
	showspaces=false,  
	showstringspaces=false, 
	showtabs=false,    
	stepnumber=1,         
	stringstyle=\color{mymauve},        % string literal style
	tabsize=4,          
	title=\lstname           
}

%\sum\nolimits_{j=1}^{M}   上下标位于求和符号的水平右端,
%\sum\limits_{j=1}^{M}   上下标位于求和符号的上下处,
%\sum_{j=1}^{M}  对上下标位置没有设定,会随公式所处环境自动调整。

%%%%%%%%%%%%%画图包%%%%%%%%%%%%%
\usepackage{tikz}
%%%%%%%%%%%%%好看的矩形%%%%%%%%%%%%%
\tikzset{
  rect1/.style = {
    shape = rectangle,% 指定样式
    minimum height=2cm,% 最小高度
    minimum width=4cm,% 最小宽度
    align = center,% 文字居中
    drop shadow,% 阴影
  }
}
%%%%%%%%%%%%%画图背景包%%%%%%%%%%%%%
\usetikzlibrary{backgrounds}

%%%%%%%%%%%%%在tikz中画一个顶点%%%%%%%%%%%%%
%%%%%%%%%%%%%#1:node名称%%%%%%%%%%%%%
%%%%%%%%%%%%%#2:位置%%%%%%%%%%%%%
%%%%%%%%%%%%%#3:标签%%%%%%%%%%%%%
\newcommand{\newVertex}[3]{\node[circle, draw=black, line width=1pt, scale=0.8] (#1) at #2{#3}}
%%%%%%%%%%%%%在tikz中画一条边%%%%%%%%%%%%%
\newcommand{\newEdge}[2]{\draw [black,very thick](#1)--(#2)}
%%%%%%%%%%%%%在tikz中放一个标签%%%%%%%%%%%%%
%%%%%%%%%%%%%#1:名称%%%%%%%%%%%%%
%%%%%%%%%%%%%#2:位置%%%%%%%%%%%%%
%%%%%%%%%%%%%#3:标签内容%%%%%%%%%%%%%
\newcommand{\newLabel}[3]{\node[line width=1pt] (#1) at #2{#3}}

%%%%%%%%%%%%%强制跳过一行%%%%%%%%%%%%%
\newcommand{\jumpLine} {\hspace*{\fill} \par}
%%%%%%%%%%%%%关键点指令,可用itemise替代%%%%%%%%%%%%%
\newcommand{\average}[1]{\left\langle #1\right\rangle }
%%%%%%%%%%%%%表格内嵌套表格%%%%%%%%%%%%%
\newcommand{\keypoint}[2]{$\bullet$\textbf{#1}\quad#2\par}
%%%%%%%%%%%%%<T>平均值表示%%%%%%%%%%%%%
\newcommand{\tabincell}[2]{\begin{tabular}{#1}#2\end{tabular}}%放在导言区
%%%%%%%%%%%%%大黑点item头%%%%%%%%%%%%%
\newcommand{\itemblt}{\item[$\bullet$]}
%%%%%%%%%%%%%大圈item头%%%%%%%%%%%%%
\newcommand{\itemc}{\item[$\circ$]}
%%%%%%%%%%%%%大星星item头%%%%%%%%%%%%%
\newcommand{\itembs}{\item[$\bigstar$]}
%%%%%%%%%%%%%右▷item头%%%%%%%%%%%%%
\newcommand{\itemrhd}{\item[$\rhd$]}
%%%%%%%%%%%%%定义为%%%%%%%%%%%%%
\newcommand{\defas}{=_{df}}
%%%%%%%%%%%%%偏导%%%%%%%%%%%%%
\newcommand{\partialx}[2]{\frac{\partial #1}{\partial #2}}
%%%%%%%%%%%%%蕴含%%%%%%%%%%%%%
\newcommand{\imp}{\rightarrow}
%%%%%%%%%%%%%上取整%%%%%%%%%%%%%
\newcommand{\ceil}[1]{\lceil#1\rceil}
%%%%%%%%%%%%%下取整%%%%%%%%%%%%%
\newcommand{\floor}[1]{\lfloor#1\rfloor}
%%%%%%%%%%%%%textbullet%%%%%%%%%%%%%
\newcommand{\blt}{\bullet}
%%%%%%%%%%%%%右箭头上加字%%%%%%%%%%%%%
\newcommand{\righttextarrow}[1]{\stackrel{#1}{\longrightarrow}}
%%%%%%%%%%%%%左箭头上加字%%%%%%%%%%%%%
\newcommand{\lefttextarrow}[1]{\stackrel{#1}{\longleftarrow}}

%%%%%%%%%%%%%双线分割线%%%%%%%%%%%%%
\newcommand*{\doublerule}{\hrule width \hsize height 1pt \kern 0.5mm \hrule width \hsize height 2pt}
%%%%%%%%%%%%%双线中间可加东西的分割线%%%%%%%%%%%%%
\newcommand\doublerulefill{\leavevmode\leaders\vbox{\hrule width .1pt\kern1pt\hrule}\hfill\kern0pt }
%%%%%%%%%%%%%左大括号%%%%%%%%%%%%%
\newcommand{\leftbig}[1]{\left\{\begin{array}{l}#1\end{array}\right.}
%%%%%%%%%%%%%矩阵%%%%%%%%%%%%%
\newcommand{\mat}[2]{\left[\begin{array}{#1}#2\end{array}\right]}
%%%%%%%%%%%%%可换行圆角文本框%%%%%%%%%%%%%
\newcommand{\ovalboxn}[1]{\ovalbox{\tabincell{l}{#1}}}
%%%%%%%%%%%%%设置section的counter, 使从1开始%%%%%%%%%%%%%
\setcounter{section}{0}

%%%%%%%%%%%%%Colors%%%%%%%%%%%%%
\newcommand{\lightercolor}[3]{% Reference Color, Percentage, New Color Name
    \colorlet{#3}{#1!#2!white}
}
\newcommand{\darkercolor}[3]{% Reference Color, Percentage, New Color Name
    \colorlet{#3}{#1!#2!black}
}
\definecolor{aquamarine}{rgb}{0.5, 1.0, 0.83}
\definecolor{Seashell}{RGB}{255, 245, 238} %背景色浅一点的
\definecolor{Firebrick4}{RGB}{255, 0, 0}%文字颜色红一点的
\lightercolor{gray}{15}{lgray}
\newcommand{\hlg}[1]{
	\begingroup
		\sethlcolor{lgray}%背景色
		\textcolor{black}{\hl{\mbox{#1}}}%textcolor里面对应文字颜色
	\endgroup
}



\title{编译原理与技术 H5}
\date{}
\author{PB18111697 王章瀚}

\begin{document}
\maketitle
\section*{4.12}
\noindent 为文法
$$\begin{array}{l}
S\rightarrow(L)|a\\
L\rightarrow L,S|S
\end{array}$$
\subsection*{(b)}
\noindent 分别写出相应的语法制导定义, 翻译方案以及预测翻译器, 它打印出每个\hlg{$a$}在句子中是第几个字符. 例如, 当句子是 \hlg{$(a,(a,(a,a),(a)))$} 时, 打印结果是 \hlg{2 5 8 10 14}\\

\noindent \textcolor{red}{消除了左递归的1,2问答案在第3问都有描述}
\subsubsection*{1. 语法制导定义}
\noindent 分析: 
\begin{itemize}
	\item 字符$a$的位置能够通过其左边部分判断, 因此使用L属性即可解决
	\item 字符$a$的位置不可能由字符$a$本身就判定, 因此需要继承属性
\end{itemize}
综上: 用$first$(继承属性)和$last$(综合属性)表示一个文法符号的第一个字符和最后一个字符的位置, 并加入一个$S'\rightarrow S$以作为初始情况.
	\begin{center}
	\begin{tabular}{l|l}
	\hline
	产生式 & 语义规则 \\
	\hline
	$S'\rightarrow S$ & \tabincell{l}{$L.first=1$} \\
	\hline
	$S\rightarrow (L)$ & \tabincell{l}{$L.first=S.first+1$\\$S.last=L.last+1$} \\
	\hline
	$S\rightarrow a$ & \tabincell{l}{$S.last=S.first$\\$print(S.first)$} \\
	\hline
	$L\rightarrow L_1,S$ & \tabincell{l}{$L_1.first=L.first$\\ $S.first=L_1.last+2$\\$L.last=S.last$} \\
	\hline
	$L\rightarrow S$ & \tabincell{l}{$S.first=L.first$\\$L.last=S.last$} \\
	\hline
	\end{tabular}
	\end{center}
\subsubsection*{2. 翻译方案}
\noindent 根据$first$和$last$的综合和继承性, 很容易写出以下翻译方案:
$$\begin{array}{l}
S'\rightarrow \{S'.first=1;\}S\\
S\rightarrow \{L.first=S.first;\} (L) \{S.last=L.last+1;\}\\
S\rightarrow a \{S.last=S.first;\ print(S.first);\}\\
L\rightarrow \{L_1.first=L.first\} L_1,\{S.first=L_1.last+2;\}S\{L.last=S.last;\}\\
L\rightarrow \{S.first=L.first;\}S\{L.last=S.last;\}
\end{array}$$
\subsubsection*{3. 预测翻译器}
\noindent 为了设计预测翻译器, 首先需要消除左递归, 使之成为LL(1)文法. 改进后文法如下:
$$\begin{array}{l}
Q\rightarrow S\\
S\rightarrow (L)\\
S\rightarrow a\\
L\rightarrow SR\\
R\rightarrow ,SR_1\\
R\rightarrow \epsilon\\
\end{array}$$
相应语法制导定义为:
\begin{center}
	\begin{tabular}{l|l}
		\hline
		产生式 & 语义规则 \\
		\hline
		$Q\rightarrow S$ & \tabincell{l}{S.first=1} \\
		\hline
		$S\rightarrow(L)$ & \tabincell{l}{L.first=S.first \\S.last=L.last+1} \\
		\hline
		$S\rightarrow a$ & \tabincell{l}{S.last=S.first \\ print(S.first)} \\
		\hline
		$L\rightarrow SR$ & \tabincell{l}{S.first=L.first \\ R.first=S.last+1 \\ L.last=R.last}\\
		\hline
		$R\rightarrow ,SR_1$ & \tabincell{l}{$S.first=R.first+1$ \\ $R_1.first=S.last+1$ \\ $R.last=R_1.last$}\\
		\hline
		$R\rightarrow\epsilon$ & \tabincell{l}{$R.last=R.fisrt$}\\
		\hline
	\end{tabular}
\end{center}
从而得到翻译方案:
$$\begin{array}{l}
Q\rightarrow \{S.first=1;\}S\\
S\rightarrow \{L.first=S.first;\} (L) \{S.last=L.last+1;\}\\
S\rightarrow a\{S.last=S.first;\ print(S.first);\}\\
L\rightarrow \{S.first=L.first\}S\{R.first=S.last+1\}R\{L.last=R.last\}\\
R\rightarrow ,\{S.first=R.first+1;\}S\{R_1.first=S.last+1;\}R_1\{R.last=R_1.last\}\\
R\rightarrow \{R.last=R.fisrt\}\\
\end{array}$$
进而能够写出其预测翻译器代码如下(\textbf{见下页}), 为了代码言简意赅, 这里没有加入错误处理:
\newpage
\begin{lstlisting}[language=c++]
// 分析 Q 的函数
void Q() {
	int s_first = 1;
	S(s_first);
}

// 分析 S 的函数
int S(int s_first) {
	int s_last;
	
	if(lookahread == 'a') { // 产生式 S->a
		match('a');
		s_last = s_first;
	}
	else { // 产生式 S->(L)
		int l_first, l_last;
		l_first = s_first;
		match('(');
		l_last = L(l_first);
		match(')');
		s_last = l_last + 1;
	}
	return s_last
}

// 分析L的函数
int L(int l_first) { //产生式 L->SR
	int l_last, s_first, s_last, r_first, r_last;
	
	s_first = l_first;
	s_last = S(s_first);
	r_first = s_last + 1;
	r_last = R(r_first);
	l_last = r_last;
	return l_last;
}

// 分析R的函数
int R(int r_first) {
	int r_last;
	
	if(lookahead == ',') { // 产生式 R->,SR1
		int s_first, s_last, r1_first, r1_last;
		s_first = r_first + 1;
		s_last = S(s_first);
		r1_first = s_last + 1;
		r1_last = R(r1_first);
		r_last = r1_last;
	}
	else {
		r_last = r_first;
	}
	return r_last;
}

\end{lstlisting}

\end{document}





