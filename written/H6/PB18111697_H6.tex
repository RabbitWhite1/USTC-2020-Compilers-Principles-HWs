\documentclass[UTF8]{article}
\usepackage{graphicx}
\usepackage{subfigure}
\usepackage{amsmath}
\usepackage{makecell}
\usepackage[utf8]{inputenc}
\usepackage[space]{ctex} %中文包
\usepackage{listings} %放代码
\usepackage{xcolor} %代码着色宏包
\usepackage{CJK} %显示中文宏包
\usepackage{float}
\usepackage{diagbox}
\usepackage{bm}
\usepackage{ulem} 
\usepackage{amssymb}
\usepackage{soul}
\usepackage{color}
\usepackage{geometry}
\usepackage{fancybox} %花里胡哨的盒子
\usepackage{xhfill} %填充包, 可画分割线 https://www.latexstudio.net/archives/8245
\usepackage{multicol} %多栏包
\usepackage{enumerate} %可以方便地自定义枚举标题
\usepackage{multirow} %表格中多行单元格合并
\usepackage{wasysym} %可以使用wasysym里的一堆奇奇怪怪的符号
\usepackage{hyperref} % url
%%%%%%%%%%%%%%%伪代码%%%%%%%%%%%%%%%
\usepackage{amsmath}
\usepackage{algorithm}
\usepackage[noend]{algpseudocode}
%%%%%%%%%%%%%%%画图包%%%%%%%%%%%%%%%
\usepackage{tikz}
\usepackage{pgfplots} % http://pgfplots.sourceforge.net/gallery.html
\usetikzlibrary{pgfplots.patchplots} % 拟合支持
\usetikzlibrary{arrows,shapes,automata,petri,positioning,calc} % 状态图支持
\usetikzlibrary{arrows.meta} % 箭头
\usetikzlibrary{shadows} % 阴影支持
\usepackage{forest} % 画树

\geometry{left = 1.5cm, right = 1.5cm, top=1.5cm, bottom=2cm}

\definecolor{mygreen}{rgb}{0,0.6,0}
\definecolor{mygray}{rgb}{0.5,0.5,0.5}
\definecolor{mymauve}{rgb}{0.58,0,0.82}
\lstset{
	backgroundcolor=\color{white}, 
	%\tiny < \scriptsize < \footnotesize < \small < \normalsize < \large < \Large < \LARGE < \huge < \Huge
	basicstyle = \footnotesize,       
	breakatwhitespace = false,        
	breaklines = true,                 
	captionpos = b,                    
	commentstyle = \color{mygreen}\bfseries,
	extendedchars = false,
	frame = shadowbox, 
	framerule=0.5pt,
	keepspaces=true,
	keywordstyle=\color{blue}\bfseries, % keyword style
	language = C++,                     % the language of code
	otherkeywords={string}, 
	numbers=left, 
	numbersep=5pt,
	numberstyle=\tiny\color{mygray},
	rulecolor=\color{black},         
	showspaces=false,  
	showstringspaces=false, 
	showtabs=false,    
	stepnumber=1,         
	stringstyle=\color{mymauve},        % string literal style
	tabsize=4,          
	title=\lstname           
}

%\sum\nolimits_{j=1}^{M}   上下标位于求和符号的水平右端,
%\sum\limits_{j=1}^{M}   上下标位于求和符号的上下处,
%\sum_{j=1}^{M}  对上下标位置没有设定,会随公式所处环境自动调整。

%%%%%%%%%%%%%画图包%%%%%%%%%%%%%
\usepackage{tikz}
%%%%%%%%%%%%%好看的矩形%%%%%%%%%%%%%
\tikzset{
  rect1/.style = {
    shape = rectangle,% 指定样式
    minimum height=2cm,% 最小高度
    minimum width=4cm,% 最小宽度
    align = center,% 文字居中
    drop shadow,% 阴影
  }
}
%%%%%%%%%%%%%画图背景包%%%%%%%%%%%%%
\usetikzlibrary{backgrounds}

%%%%%%%%%%%%%在tikz中画一个顶点%%%%%%%%%%%%%
%%%%%%%%%%%%%#1:node名称%%%%%%%%%%%%%
%%%%%%%%%%%%%#2:位置%%%%%%%%%%%%%
%%%%%%%%%%%%%#3:标签%%%%%%%%%%%%%
\newcommand{\newVertex}[3]{\node[circle, draw=black, line width=1pt, scale=0.8] (#1) at #2{#3}}
%%%%%%%%%%%%%在tikz中画一条边%%%%%%%%%%%%%
\newcommand{\newEdge}[2]{\draw [black,very thick](#1)--(#2)}
%%%%%%%%%%%%%在tikz中放一个标签%%%%%%%%%%%%%
%%%%%%%%%%%%%#1:名称%%%%%%%%%%%%%
%%%%%%%%%%%%%#2:位置%%%%%%%%%%%%%
%%%%%%%%%%%%%#3:标签内容%%%%%%%%%%%%%
\newcommand{\newLabel}[3]{\node[line width=1pt] (#1) at #2{#3}}

%%%%%%%%%%%%%强制跳过一行%%%%%%%%%%%%%
\newcommand{\jumpLine} {\hspace*{\fill} \par}
%%%%%%%%%%%%%关键点指令,可用itemise替代%%%%%%%%%%%%%
\newcommand{\average}[1]{\left\langle #1\right\rangle }
%%%%%%%%%%%%%表格内嵌套表格%%%%%%%%%%%%%
\newcommand{\keypoint}[2]{$\bullet$\textbf{#1}\quad#2\par}
%%%%%%%%%%%%%<T>平均值表示%%%%%%%%%%%%%
\newcommand{\tabincell}[2]{\begin{tabular}{#1}#2\end{tabular}}%放在导言区
%%%%%%%%%%%%%大黑点item头%%%%%%%%%%%%%
\newcommand{\itemblt}{\item[$\bullet$]}
%%%%%%%%%%%%%大圈item头%%%%%%%%%%%%%
\newcommand{\itemc}{\item[$\circ$]}
%%%%%%%%%%%%%大星星item头%%%%%%%%%%%%%
\newcommand{\itembs}{\item[$\bigstar$]}
%%%%%%%%%%%%%右▷item头%%%%%%%%%%%%%
\newcommand{\itemrhd}{\item[$\rhd$]}
%%%%%%%%%%%%%定义为%%%%%%%%%%%%%
\newcommand{\defas}{=_{df}}
%%%%%%%%%%%%%偏导%%%%%%%%%%%%%
\newcommand{\partialx}[2]{\frac{\partial #1}{\partial #2}}
%%%%%%%%%%%%%蕴含%%%%%%%%%%%%%
\newcommand{\imp}{\rightarrow}
%%%%%%%%%%%%%上取整%%%%%%%%%%%%%
\newcommand{\ceil}[1]{\lceil#1\rceil}
%%%%%%%%%%%%%下取整%%%%%%%%%%%%%
\newcommand{\floor}[1]{\lfloor#1\rfloor}
%%%%%%%%%%%%%textbullet%%%%%%%%%%%%%
\newcommand{\blt}{\bullet}
%%%%%%%%%%%%%右箭头上加字%%%%%%%%%%%%%
\newcommand{\righttextarrow}[1]{\stackrel{#1}{\longrightarrow}}
%%%%%%%%%%%%%左箭头上加字%%%%%%%%%%%%%
\newcommand{\lefttextarrow}[1]{\stackrel{#1}{\longleftarrow}}

%%%%%%%%%%%%%双线分割线%%%%%%%%%%%%%
\newcommand*{\doublerule}{\hrule width \hsize height 1pt \kern 0.5mm \hrule width \hsize height 2pt}
%%%%%%%%%%%%%双线中间可加东西的分割线%%%%%%%%%%%%%
\newcommand\doublerulefill{\leavevmode\leaders\vbox{\hrule width .1pt\kern1pt\hrule}\hfill\kern0pt }
%%%%%%%%%%%%%左大括号%%%%%%%%%%%%%
\newcommand{\leftbig}[1]{\left\{\begin{array}{l}#1\end{array}\right.}
%%%%%%%%%%%%%矩阵%%%%%%%%%%%%%
\newcommand{\mat}[2]{\left[\begin{array}{#1}#2\end{array}\right]}
%%%%%%%%%%%%%可换行圆角文本框%%%%%%%%%%%%%
\newcommand{\ovalboxn}[1]{\ovalbox{\tabincell{l}{#1}}}
%%%%%%%%%%%%%设置section的counter, 使从1开始%%%%%%%%%%%%%
\setcounter{section}{0}

%%%%%%%%%%%%%Colors%%%%%%%%%%%%%
\newcommand{\lightercolor}[3]{% Reference Color, Percentage, New Color Name
    \colorlet{#3}{#1!#2!white}
}
\newcommand{\darkercolor}[3]{% Reference Color, Percentage, New Color Name
    \colorlet{#3}{#1!#2!black}
}
\definecolor{aquamarine}{rgb}{0.5, 1.0, 0.83}
\definecolor{Seashell}{RGB}{255, 245, 238} %背景色浅一点的
\definecolor{Firebrick4}{RGB}{255, 0, 0}%文字颜色红一点的
\lightercolor{gray}{15}{lgray}
\newcommand{\hlg}[1]{
	\begingroup
		\sethlcolor{lgray}%背景色
		\textcolor{black}{\hl{\mbox{#1}}}%textcolor里面对应文字颜色
	\endgroup
}



\title{编译原理与技术 H6}
\date{}
\author{PB18111697 王章瀚}

\begin{document}
\maketitle
\section*{4.12}
\noindent 为文法
$$\begin{array}{l}
S\rightarrow(L)|a\\
L\rightarrow L,S|S
\end{array}$$
(b) 写出自上而下分析的栈操作代码, 它打印每个\hlg{$a$}在句子中是第几个字符, 例如, 当句子是\hlg{$(a,(a,(a,a),(a)))$}时, 打印的结果是\hlg{$2\quad5\quad8\quad10\quad14$}\\

上一次作业中已经写出了消除了左递归的文法和翻译方案如下:\\
\begin{minipage}{\linewidth/3}
$$\begin{array}{l}
	Q\rightarrow S\\
	S\rightarrow (L)\\
	S\rightarrow a\\
	L\rightarrow SR\\
	R\rightarrow ,SR_1\\
	R\rightarrow \epsilon\\
\end{array}$$
\end{minipage}
\begin{minipage}{\linewidth*2/3}
$$\begin{array}{l}
	Q\rightarrow \{S.first=1;\}S\\
	S\rightarrow \{L.first=S.first;\} (L) \{S.last=L.last+1;\}\\
	S\rightarrow a\{S.last=S.first;\ print(S.first);\}\\
	L\rightarrow \{S.first=L.first\}S\{R.first=S.last+1\}R\{L.last=R.last\}\\
	R\rightarrow ,\{S.first=R.first+1;\}S\{R_1.first=S.last+1;\}R_1\{R.last=R_1.last\}\\
	R\rightarrow \{R.last=R.fisrt\}\\
\end{array}$$
\end{minipage}

这样, 对于一个LL(1)文法来说, 只需要系统地引入非终结符即可在LR分析期间完成属性计算. 引入的符号及引入后对栈操作代码如下:\\
\begin{center}
\begin{tabular}{|l|l|l|}
	\hline
	翻译方案 & 语义规则 & 操作代码 \\
	\hline
	$Q\rightarrow AS$ & $S.first=A.last$ & $val[top-1]=val[top]$\\
	\hline
	$A\rightarrow \epsilon$ & $A.last=1$ & $val[top]=1$\\
	\hline
	$S\rightarrow (BL)$ & $B.first=S.first+1$ & $val[top-3]=val[top-1]+1$\\
	& $L.first=B.last+1$ & \\
	& $S.last=L.last+1$ & \\
	\hline
	$B\rightarrow\epsilon$ & $B.last=B.first$ & $val[top+1]=val[top-1]+2$ \\
	\hline
	$S\rightarrow a$ & $S.last=S.first; print(S.first)$ & $val[top]=val[top-1]+1$\\
	\hline
	$L\rightarrow SR$ & $S.first=L.first+1$ & $val[top-1]=val[top]$ \\
	& $R.first=S.last+1$ & \\
	& $L.last=R.last$ & \\
	\hline
	$R\rightarrow ,DSR_1$ & $D.first=R.first+1$ & $val[top-3]=val[top]$ \\
	& $S.first=D.last+1$ & \\
	& $R_1.first=S.last+1$ & \\
	& $L.last=R_1.last$ & \\
	\hline
	$D\rightarrow \epsilon$ & $D.last=L.first$ & $val[top+1]=val[top-1]+1$ \\
	\hline
	$R\rightarrow \epsilon$ & $R.last=R.first$ & $val[top+1]=val[top]$ \\
	\hline
\end{tabular}
\end{center}



\newpage
\section*{4.14}
\noindent 程序的文法如下:
$$\begin{array}{l}
	P\rightarrow D\\
	D\rightarrow D;D | id:T | proc\ id; D;S
\end{array}$$
\begin{enumerate}[(a). ]
\item 写一个语法制导定义, 打印该程序一共声明了多少个id.\\
即考虑文法(这里的 \hlg{$T$} 和 \hlg{$S$} 就当成终结符了)
$$\begin{array}{l}
P\rightarrow D \\
D\rightarrow D;D \\
D\rightarrow id:T \\
D\rightarrow proc\ id; D;S
\end{array}$$
用 \hlg{$count$} 表示直到当前符号包含的 \hlg{$id$} 个数. 则其语法制导定义如下:\\
\begin{center}
\begin{tabular}{l|l}
	\hline
	产生式 & 语义规则 \\
	\hline
	$P\rightarrow D$ & $print(D.count)$ \\
	\hline
	$D\rightarrow D_1;D_2$ & $D.count=D_1.count+D_2.count$ \\
	\hline
	$D\rightarrow id:T$ & $D.count=1$ \\
	\hline
	$D\rightarrow proc\ id; D_1;S$ & $D.count=1+D_1.count$ \\
	\hline
\end{tabular}
\end{center}
\item 写一个翻译方案, 打印该程序每个变量id的嵌套深度. 例如, 当句型是 \hlg{$a:T;proc\ b;ba:T;S$}时, \hlg{$a$}和\hlg{$b$}的嵌套深度是1, \hlg{$ba$}的嵌套深度是2.\\
为了表示嵌套深度显然需要引入继承属性. 这里用 \hlg{$depth$}表示嵌套深度. 于是可以写出翻译方案如下:
$$\begin{array}{l}
P\rightarrow \{D.depth=0;\}D \\
D\rightarrow \{D_1.depth=D.depth+1;\}D_1;\{D_2.depth=D.depth+1;\}D_2 \\
D\rightarrow id:T \{print(D.depth);\}\\
D\rightarrow proc\ id\{print(D.depth);\}; \{D_1.depth=D.depth+1;\}D_1;S
\end{array}$$

\end{enumerate}




\end{document}





