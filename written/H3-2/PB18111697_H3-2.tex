\documentclass[UTF8]{article}
\usepackage{graphicx}
\usepackage{subfigure}
\usepackage{amsmath}
\usepackage{makecell}
\usepackage[utf8]{inputenc}
\usepackage[space]{ctex} %中文包
\usepackage{listings} %放代码
\usepackage{xcolor} %代码着色宏包
\usepackage{CJK} %显示中文宏包
\usepackage{float}
\usepackage{makecell}
\usepackage{diagbox}
\usepackage{bm}
\usepackage{ulem} 
\usepackage{amssymb}
\usepackage{soul}
\usepackage{color}
\usepackage{geometry}
\usepackage{fancybox} %花里胡哨的盒子
\usepackage{xhfill} %填充包, 可画分割线 https://www.latexstudio.net/archives/8245
\usepackage{multicol} %多栏包
\usepackage{enumerate} %可以方便地自定义枚举标题
\usepackage{multirow} %表格中多行单元格合并
\usepackage{wasysym} %可以使用wasysym里的一堆奇奇怪怪的符号
%%%%%%%%%%%%%%%伪代码%%%%%%%%%%%%%%%
\usepackage{amsmath}
\usepackage{algorithm}
\usepackage[noend]{algpseudocode}
%%%%%%%%%%%%%%%画图包%%%%%%%%%%%%%%%
\usepackage{tikz}
\usepackage{tikZ-timing} % 时序图支持
\usepackage{pgfplots} % http://pgfplots.sourceforge.net/gallery.html
\usetikzlibrary{pgfplots.patchplots} % 拟合支持
\usetikzlibrary{arrows,shapes,automata,petri,positioning,calc} % 状态图支持
\usetikzlibrary{shadows} % 阴影支持
\usepackage{forest} % 画树

\geometry{left = 1.5cm, right = 1.5cm, top=1.5cm, bottom=2cm}

\definecolor{mygreen}{rgb}{0,0.6,0}
\definecolor{mygray}{rgb}{0.5,0.5,0.5}
\definecolor{mymauve}{rgb}{0.58,0,0.82}
\lstset{
	backgroundcolor=\color{white}, 
	%\tiny < \scriptsize < \footnotesize < \small < \normalsize < \large < \Large < \LARGE < \huge < \Huge
	basicstyle = \footnotesize,       
	breakatwhitespace = false,        
	breaklines = true,                 
	captionpos = b,                    
	commentstyle = \color{mygreen}\bfseries,
	extendedchars = false,
	frame = shadowbox, 
	framerule=0.5pt,
	keepspaces=true,
	keywordstyle=\color{blue}\bfseries, % keyword style
	language = C++,                     % the language of code
	otherkeywords={string}, 
	numbers=left, 
	numbersep=5pt,
	numberstyle=\tiny\color{mygray},
	rulecolor=\color{black},         
	showspaces=false,  
	showstringspaces=false, 
	showtabs=false,    
	stepnumber=1,         
	stringstyle=\color{mymauve},        % string literal style
	tabsize=4,          
	title=\lstname           
}

%\sum\nolimits_{j=1}^{M}   上下标位于求和符号的水平右端,
%\sum\limits_{j=1}^{M}   上下标位于求和符号的上下处,
%\sum_{j=1}^{M}  对上下标位置没有设定,会随公式所处环境自动调整。

%%%%%%%%%%%%%画图包%%%%%%%%%%%%%
\usepackage{tikz}
%%%%%%%%%%%%%好看的矩形%%%%%%%%%%%%%
\tikzset{
  rect1/.style = {
    shape = rectangle,% 指定样式
    minimum height=2cm,% 最小高度
    minimum width=4cm,% 最小宽度
    align = center,% 文字居中
    drop shadow,% 阴影
  }
}
%%%%%%%%%%%%%画图背景包%%%%%%%%%%%%%
\usetikzlibrary{backgrounds}

%%%%%%%%%%%%%在tikz中画一个顶点%%%%%%%%%%%%%
%%%%%%%%%%%%%#1:node名称%%%%%%%%%%%%%
%%%%%%%%%%%%%#2:位置%%%%%%%%%%%%%
%%%%%%%%%%%%%#3:标签%%%%%%%%%%%%%
\newcommand{\newVertex}[3]{\node[circle, draw=black, line width=1pt, scale=0.8] (#1) at #2{#3}}
%%%%%%%%%%%%%在tikz中画一条边%%%%%%%%%%%%%
\newcommand{\newEdge}[2]{\draw [black,very thick](#1)--(#2)}
%%%%%%%%%%%%%在tikz中放一个标签%%%%%%%%%%%%%
%%%%%%%%%%%%%#1:名称%%%%%%%%%%%%%
%%%%%%%%%%%%%#2:位置%%%%%%%%%%%%%
%%%%%%%%%%%%%#3:标签内容%%%%%%%%%%%%%
\newcommand{\newLabel}[3]{\node[line width=1pt] (#1) at #2{#3}}

%%%%%%%%%%%%%强制跳过一行%%%%%%%%%%%%%
\newcommand{\jumpLine} {\hspace*{\fill} \par}
%%%%%%%%%%%%%关键点指令,可用itemise替代%%%%%%%%%%%%%
\newcommand{\average}[1]{\left\langle #1\right\rangle }
%%%%%%%%%%%%%表格内嵌套表格%%%%%%%%%%%%%
\newcommand{\keypoint}[2]{$\bullet$\textbf{#1}\quad#2\par}
%%%%%%%%%%%%%<T>平均值表示%%%%%%%%%%%%%
\newcommand{\tabincell}[2]{\begin{tabular}{@{}#1@{}}#2\end{tabular}}%放在导言区
%%%%%%%%%%%%%大黑点item头%%%%%%%%%%%%%
\newcommand{\itemblt}{\item[$\bullet$]}
%%%%%%%%%%%%%大圈item头%%%%%%%%%%%%%
\newcommand{\itemc}{\item[$\circ$]}
%%%%%%%%%%%%%大星星item头%%%%%%%%%%%%%
\newcommand{\itembs}{\item[$\bigstar$]}
%%%%%%%%%%%%%右▷item头%%%%%%%%%%%%%
\newcommand{\itemrhd}{\item[$\rhd$]}
%%%%%%%%%%%%%定义为%%%%%%%%%%%%%
\newcommand{\defas}{=_{df}}
%%%%%%%%%%%%%蕴含%%%%%%%%%%%%%
\newcommand{\imp}{\rightarrow}

%%%%%%%%%%%%%双线分割线%%%%%%%%%%%%%
\newcommand*{\doublerule}{\hrule width \hsize height 1pt \kern 0.5mm \hrule width \hsize height 2pt}
%%%%%%%%%%%%%双线中间可加东西的分割线%%%%%%%%%%%%%
\newcommand\doublerulefill{\leavevmode\leaders\vbox{\hrule width .1pt\kern1pt\hrule}\hfill\kern0pt }
%%%%%%%%%%%%%左大括号%%%%%%%%%%%%%
\newcommand{\leftbig}[1]{\left\{\begin{array}{l}#1\end{array}\right.}
%%%%%%%%%%%%%矩阵%%%%%%%%%%%%%
\newcommand{\mat}[2]{\left[\begin{array}{#1}#2\end{array}\right]}
%%%%%%%%%%%%%可换行圆角文本框%%%%%%%%%%%%%
\newcommand{\ovalboxn}[1]{\ovalbox{\tabincell{l}{#1}}}
%%%%%%%%%%%%%设置section的counter, 使从1开始%%%%%%%%%%%%%
\setcounter{section}{0}

%%%%%%%%%%%%%Colors%%%%%%%%%%%%%
\definecolor{aquamarine}{rgb}{0.5, 1.0, 0.83}
\newcommand{\red}[1]{\textcolor{red}{#1}}



\title{编译原理与技术 H3-2}
\date{}
\author{PB18111697 王章瀚}

\begin{document}
\maketitle
\section*{3.16}

\begin{enumerate}[(a) ]
\item 用习题3.1的文法(如下)构造$(a, (a, a))$的最右推导, 说出每个右句型的句柄\\
$S\rightarrow (L)|a$\\$L\rightarrow L,S|S$\\
	最右推导为: $$S\rightarrow(L)\rightarrow(L,S)\rightarrow(L,(L))\rightarrow(L,(L,S)),\rightarrow(L,(L,a))\rightarrow(L,(S,a))\rightarrow(L,(a,a))\rightarrow(S,(a,a))\rightarrow(a,(a,a))$$
	用红色标记出句柄如下:
	$$S\rightarrow\red{(L)}\rightarrow(\red{L,S})\rightarrow(L,\red{(L)})\rightarrow(L,(\red{L,S})),\rightarrow(L,(L,\red{a}))\rightarrow(L,(\red{S},a))\rightarrow(L,(\red{a},a))\rightarrow(\red{S},(a,a))\rightarrow(\red{a},(a,a))$$
	为方便对照, 对应句柄列表如下:\\
	\begin{tabular}{|c|c|}
	\hline
	右句型 & 句柄 \\
	\hline
	$(L)$ & $(L)$ \\
	\hline
	$(L,S)$ & $L,S$ \\
	\hline
	$(L,(L))$ & $(L)$ \\
	\hline
	$(L,(L,S))$ & $L,S$ \\
	\hline
	$(L,(L,a))$ & $a$ \\
	\hline
	$(L,(S,a))$ & $S$ \\
	\hline
	$(L,(a,a))$ & $a$ \\
	\hline
	$(S,(a,a))$ & $S$ \\
	\hline
	$(a,(a,a))$ & $a$ \\
	\hline
	\end{tabular}
	
\newpage
\item 给出对应(a)的最右推导的移进-归约分析器的步骤\\
	\begin{tabular}{c|c|c}
	\hline
	栈 & 输入 & 动作 \\
	\hline
	$\$$ & $(a,(a,a))\$$ & 移进 \\
	\hline
	$\$($ & $a,(a,a))\$$ & 移进 \\ 
	\hline
	$\$(a$ & $,(a,a))\$$ & 按$S\rightarrow a$归约 \\
	\hline
	$\$(S$ & $,(a,a))\$$ & 按$L\rightarrow S$归约 \\
	\hline
	$\$(L$ & $,(a,a))\$$ & 移进 \\
	\hline
	$\$(L,$ & $(a,a))\$$ & 移进 \\
	\hline
	$\$(L,($ & $a,a))\$$ & 移进 \\
	\hline
	$\$(L,(a$ & $,a))\$$ & 按$S\rightarrow a$归约 \\
	\hline
	$\$(L,(S$ & $,a))\$$ & 按$L\rightarrow S$归约 \\
	\hline
	$\$(L,(L$ & $,a))\$$ & 移进 \\
	\hline
	$\$(L,(L,$ & $a))\$$ & 移进 \\
	\hline
	$\$(L,(L,a$ & $))\$$ & 按$S\rightarrow a$归约 \\
	\hline
	$\$(L,(L,S$ & $))\$$ & 按$L\rightarrow L,S$归约 \\
	\hline
	$\$(L,(L$ & $))\$$ & 移进 \\
	\hline
	$\$(L,(L)$ & $)\$$ & 按$S\rightarrow (L)$归约 \\
	\hline
	$\$(L,S$ & $)\$$ & 按$L\rightarrow L,S$归约 \\
	\hline
	$\$(L$ & $)\$$ & 移进 \\
	\hline
	$\$(L)$ & $\$$ & 按$S\rightarrow (L)$归约 \\
	\hline
	$\$S$ & $\$$ & 接受 \\
	\hline
	\end{tabular}
	\newpage
\item 对照(b)的移进-规约, 给出自下而上的构造分析树的步骤\\
	如下:\\
	\begin{forest}
	where n children=0{tier=word}{}
	[$S$, before drawing tree={x-=15mm, y+=15mm}, for tree={s sep=5mm, l sep=(6-level)*4mm}
		[$($, before drawing tree={x-=15mm}]
		[$L$
			[$L$[$S$[$a$]]]
			[,]
			[$S$
				[$($]
				[$L$
					[$L$[$S$[$a$]]]
					[,]
					[$S$[$a$]]
				]
				[$)$]
			]
		]
		[$)$, before drawing tree={x+=0mm}]
		[$\$$]
	]
	\end{forest}
\end{enumerate}

\end{document}







